\section{Statistics and experiments}

\note{TODO: Expand this sketch}

Some statistics are discussed \href{https://leanprover.zulipchat.com/#narrow/channel/458659-Equational/topic/A.20final.20end-to-end.20theorem.20in.20Lean}{here}.

Data analysis of the implication graph:

\begin{itemize}
    \item Mention the long chain $2 \Rightarrow 5 \Rightarrow 2499 \Rightarrow 2415 \Rightarrow 238 \Rightarrow 2716 \Rightarrow 28 \Rightarrow 2973 \Rightarrow 270 \Rightarrow 3 \Rightarrow 3715 \Rightarrow 375 \Rightarrow 359 \Rightarrow 4065 \Rightarrow 1$ (discussed \href{https://leanprover.zulipchat.com/#narrow/stream/458659-Equational/topic/visualization.20of.20graph.20closure}{here}).
    \item What are the "most difficult" implications?
    \item Is there a way to generate a standard test set of implication problems of various difficulty levels? Can one then use this to benchmark various automated and semi-automated methods? Challenge: how does one automatically assign a difficulty level to a given (anti-)implication?
\end{itemize}


See \href{https://leanprover.zulipchat.com/#narrow/channel/458659-Equational/topic/Outlier.20hunting}{this} for a preliminary data analysis of the impact of equation size and the number of variables.

Analyze the implication graph and discuss test sets of implication problems for benchmarking theorem provers. Challenge: How can one automatically assign a difficulty level to an implication?
